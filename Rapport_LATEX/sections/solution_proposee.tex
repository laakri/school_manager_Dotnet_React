\section{Solution Proposée}
\label{sec:solution_proposee}

Pour répondre aux besoins identifiés, le Student Absence Tracker propose une solution moderne combinant plusieurs technologies :

\subsection{Architecture Globale}
L'application est conçue selon une architecture client-serveur :

\begin{itemize}
    \item \textbf{Frontend} : Interface utilisateur développée avec React et TypeScript
    \item \textbf{Backend} : API REST construite avec ASP.NET Core
    \item \textbf{Base de données} : SQL Server pour le stockage persistant
    \item \textbf{Authentication} : Système d'authentification intégré
\end{itemize}

\subsection{Composants Clés}

\subsubsection{Interface Administrative}
Interface complète pour la gestion des données de base :
\begin{itemize}
    \item Gestion des enseignants
    \item Gestion des étudiants
    \item Configuration des classes
    \item Définition des matières
\end{itemize}

\subsubsection{Système de Suivi des Absences}
\begin{itemize}
    \item Interface de saisie rapide des absences
    \item Visualisation des données par classe/matière
    \item Génération de rapports
    \item Historique détaillé
\end{itemize}

\subsection{Avantages de la Solution}

\begin{itemize}
    \item \textbf{Efficacité} : Processus de suivi des absences simplifié
    \item \textbf{Évolutivité} : Architecture modulaire extensible
    \item \textbf{Performance} : Technologies modernes et optimisées
    \item \textbf{Sécurité} : Authentification et autorisation robustes
    \item \textbf{Maintenance} : Code structuré et bien documenté
\end{itemize}

