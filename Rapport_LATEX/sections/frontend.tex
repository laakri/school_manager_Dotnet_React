\section{Implémentation Frontend}
\label{sec:frontend}

Le frontend du Student Absence Tracker est développé avec React et TypeScript, offrant une interface utilisateur moderne et responsive.

\subsection{Structure des Composants}
\begin{itemize}
    \item \textbf{Layout} : Structure globale de l'application
    \begin{itemize}
        \item Header : Barre de navigation supérieure
        \item Sidebar : Menu latéral pour la navigation
        \item Main Content : Zone principale de contenu
    \end{itemize}
    \item \textbf{Pages} : Composants principaux
    \begin{itemize}
        \item Dashboard : Vue d'ensemble
        \item Teachers : Gestion des enseignants
        \item Students : Gestion des étudiants
        \item Classes : Gestion des classes
        \item Subjects : Gestion des matières
        \item MarkAbsences : Saisie des absences
    \end{itemize}
\end{itemize}

\subsection{Composants Réutilisables}
\begin{itemize}
    \item \texttt{Button} : Boutons stylisés avec états
    \item \texttt{Input} : Champs de saisie avec validation
    \item \texttt{Select} : Listes déroulantes
    \item \texttt{Table} : Tableaux de données
    \item \texttt{Modal} : Fenêtres modales
    \item \texttt{DatePicker} : Sélecteur de date
\end{itemize}

\subsection{Gestion d'État}
\begin{itemize}
    \item Context API pour l'état global
    \item Hooks personnalisés pour la logique réutilisable
    \item État local avec useState pour les composants
\end{itemize}

\subsection{Communication avec l'API}
\begin{itemize}
    \item Service API centralisé
    \item Gestion des requêtes HTTP avec Fetch
    \item Gestion des erreurs et chargement
    \item Types TypeScript pour la sécurité des données
\end{itemize}
