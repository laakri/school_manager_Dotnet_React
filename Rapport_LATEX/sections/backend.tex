\section{Implémentation du Backend}
\label{sec:backend}

Le backend du Student Absence Tracker, développé avec .NET Core, constitue le cœur du système de gestion des absences.

\subsection{Structure de l'API}
L'API REST est organisée selon une architecture modulaire :

\subsubsection{Contrôleurs Principaux}
\begin{itemize}
    \item \texttt{TeachersController} : Gestion des enseignants et authentification
    \item \texttt{StudentsController} : Gestion des étudiants
    \item \texttt{ClassesController} : Gestion des classes
    \item \texttt{SubjectsController} : Gestion des matières
    \item \texttt{AbsencesController} : Gestion des absences
\end{itemize}

\subsection{Modèles de Données}
\begin{itemize}
    \item \texttt{Teacher} : Informations sur les enseignants
    \item \texttt{Student} : Données des étudiants
    \item \texttt{Class} : Configuration des classes
    \item \texttt{Subject} : Définition des matières
    \item \texttt{Absence} : Enregistrement des absences
    \item \texttt{TeacherSubjectClass} : Relations entre enseignants, matières et classes
\end{itemize}

\subsection{Entity Framework Core}
\begin{itemize}
    \item Configuration du contexte de base de données
    \item Migrations pour la gestion du schéma
    \item Relations entre les entités
    \item Requêtes optimisées avec LINQ
\end{itemize}

\subsection{Sécurité et Performance}
\begin{itemize}
    \item Authentification basée sur les noms d'utilisateur
    \item Validation des données entrantes
    \item Gestion des erreurs globale
    \item Optimisation des requêtes
\end{itemize}
